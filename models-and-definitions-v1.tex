\subsection{Model and Definitions}\label{model}

A shared atomic storage (or atomic KV-store) can be emulated
by composing individual atomic objects. Therefore, we aim
to implement \nn{a single} atomic read/write memory object. %on a set of servers.
A read/write object takes a value from a set $\valSet$. 
We assume a system consisting of three distinct sets of processes: 
a set $\wSet$ of writers, a set $\rdSet$ of readers and  $\srvSet$, a set of servers. 
 Servers host data elements (replicas or encoded data fragments).
Each writer is allowed to modify the value of a shared object, and each reader is allowed to obtain 
the value of that object. 
Nodes communicate through 
\myemph{asynchronous}, but \myemph{reliable} channels. 
%%Let $\idSet = \cSet\cup\reconSet\cup\srvSet$. 
%In a read/write object implementation, we assume that the object may take a value from a set $\valSet$. 
%Each writer is allowed to modify the value of the object, and each reader is allowed to obtain 
%the value of the object. Servers host data elements (replicas or encoded data fragments).
%%maintain encoded elements of the redundant object.
%
%We assume an \myemph{asynchronous} environment, where processes communicate
%by exchanging messages. The writer, any subset of readers, and up to 
%$f$ servers may \myemph{crash} without any notice.
%\ares{} allows the set of server host to be modified during the course of an execution for 
%masking transient or permanent failures of servers and preserve the longevity of the service.  
%\kmk{In the paper, we are interested only  in the fair executions of any algorithm.}


%we can formally define a quorum system $\quorums{c}$, with $c\in\confSet$, as follows:
%Consider a configuration identifier $c$.
%We define as $\servers{c}=\bigcup_{Q\in\quorums{c}} Q$ the set of servers that belong in the quorums
%f a quorum system $\quorums{c}$. 
%We refer to a server $s$ as a \myemph{member} of a configuration $c$ if $s\in \servers{c}$.

%
%\paragraph{Notations and Definitions}  We denote by $\mathcal{W}$, $\mathcal{R}$ and $\mathcal{S}$,  the set of writers, readers and servers, respectively.
%We denote by $\confSet$ the set of possible configurations of the systems. A configuration $c \in \confSet$ has a unique the configuration identifier   $c.conf$; a set of servers $c.servers$, such that, $c.Servers \subseteq \mathcal{S}$,
%$cseq[i].status\in\{F,P\}$, the pending or finalized status 

%The writer process $w$ may invoke write operations on the atomic object 
%by calling the $\act{write}(v)$ function whereas each reader may invoke 
%a read operation by calling the $\act{read}$ function. Each write event 
%returns an acknowledgement when successfully carried-out, whereas each read 
%operation returns the value of the atomic object. We assume that an 
%

\myparagraph{Executions.} 
%An algorithm $A$ is a collection of processes, where process $A_p$
%is assigned to processor $p\in\idSet\cup\srvSet$. The \textit{state}, of a process $A_\pr$ is determined over a
%set of state variables, and the state $\state$ of $A$ is a vector that contains the state of
%each process. Each process $A_\pr$ implements a set of actions. When an action $\acts{}$ occurs 
%it causes the state of $A_\pr$ to change, say from 
%some state $\state_p$ to some different state $\state_p'$. We call the triple $\tup{\state_p, \acts{}, \state_p'}$
%as a \textit{step} of $A_\pr$. Algorithm $A$ performs a step, when some process $A_\pr$ performs a step.
%: (i) receives a
%message, (ii) performs local computation, (iii) sends a message. 
%Each such action
%causes the state at $p$ to change. 
An \textit{execution} of an algorithm $A$  is an alternating sequence of states
and actions of $A$ starting with the initial state. 
An execution $\EX$ is \textit{well-formed} if any process invokes one operation
at a time and it is \textit{fair} if enabled actions perform a step infinitely often. In the rest of the paper 
we consider executions that are fair and well-formed. A node \textit{crashes} in an execution if it stops taking steps; otherwise it is  \textit{non-faulty}.


\myparagraph{Write and Read Operations.}
An implementation of a read or a write operation contains an \textit{invocation} action  and a \textit{response} action (such as a
	return from the procedure). An operation $\op$ is \textit{complete} in an execution, if it
	contains both the invocation and the \textit{matching} response actions for $\op$; otherwise $\op$
	is \textit{incomplete}. 
	We say that an operation $\op$ \textit{precedes} an operation $\op'$ in an execution $\EX$,
	denoted by $\op\bef\op'$, if the response step of $\op$ appears before the invocation
	step of $\op'$ in $\EX$. Two operations are \textit{concurrent} if neither precedes the other.
	An implementation $A$ of a read/write object satisfies the atomicity property
	if the conditions of Lemma 13.16 in  \cite{Lynch1996} holds. 
	
\myparagraph{Storage and Communication Costs.} 
\remove{We are interested in the \myemph{complexity} of each
read and write operation. The complexity of each operation $\op$ invoked by a process 
$\pr$, is measured with respect to three metrics, during the interval between the invocation 
and the response of $\op$: (i) \myemph{communication round-trips}, accounting the number of messages 
sent during $\op$, (ii) \myemph{storage efficiency} \nn{(storage cost)}, accounting the maximum storage requirements for 
any single object at the servers during $\op$, and  (iii) \myemph{message bit complexity} \nn{(communication cost)}
which measures the size of the messages used during $\op$. 
}
We define the total storage cost as the size of the
data stored \textit{across all servers}, at any point during the execution of the algorithm. The
communication cost associated with a read or write operation is the size of the total data transmitted in the messages sent as part of the operation. We assume that metadata,
such as version number, process ID, etc. used by various operations is of negligible size, and therefore,  ignore in the calculation of storage and communication cost. Further, we normalize
both the costs with respect to the size of the value v; in other words, we compute the costs
assuming that v has size 1 unit.
%\lewis{This is repetitive? Didn't we already discuss this before? Second paragraph of this section.}
%\blue{Yes, now removed}

 
\myparagraph{Quorum system}. We define our quorum system, $\mathcal{Q}$, to be the set of all subsets of $\mathcal{S}$ that have at least $\frac{n+k}{2}$ servers. We refer to the members of $\mathcal{Q}$, as quorum sets and they satisfy the following property.

\begin{lemma}
For any $k$,  $1 \leq k \leq n -2f$. (i) If $Q_1$ ,$Q_2 \in \mathcal{Q}$,
then $|Q_1 \cap Q_2 | \geq k$. (ii) If the number of faulty servers is
at most $f$ , then $Q$ contains at least one quorum set $Q$ of
non-faulty servers.
\end{lemma}
%\myparagraph{Storage and Communication Cost.} We define the total storage cost as the size of the
%data stored across all servers, at any point during the execution of the algorithm. The
%communication cost associated with a read or write operation is the size of the total data that
%gets transmitted in the messages sent as part of the operation. We assume that metadata,
%such as version number, process ID, etc. used by various operations is of negligible size, and
%is hence ignored in the calculation of storage and communication cost. Further, we normalize
%both the costs with respect to the size of the value v; in other words, we compute the costs
%under the assumption that v has size 1 unit.

%A communication round-trip
%(or simply round) is more formally defined in the following definition that appeared in \cite{CDGL04,GNS09,GNS08}:

%\begin{definition}\label{def:com}
%Process $p$ performs a communication round during operation $\op$ 
%if all of the following hold:
%\begin{enumerate}
%   \item $p$ sends request messages that are a part of $\op$ 
%         to a set of processes,
%	\item any process $q$ that receives a request message from $p$ for 
%         operation $\op$, replies
%% to it with an acknowledgment 
%%         message, at the first possible convenience and 
%without delay, i.e. without waiting for any other messages before 
%replying to $\op$.
%   \item when process $p$ receives ``enough'' replies it 
%terminates the round
%%, orreturns or repeats a communication round
%         %then makes a local decision about termination of the phase.
%\end{enumerate}
%\end{definition}
%
%At the end of a communication round process $\pr$ may complete $\op$
%or start a new round. Operation $\op$ is \myemph{fast} \cite{CDGL04} if it completes after 
%its first communication round; an implementation is fast if in each execution
%all operations are fast.

\myparagraph{Liveness of operations.} We require algorithms to satisfy certain liveness properties, specifically, in every fair execution that satisfies certain restrictions in terms of the number of failed nodes, we require every operation by a non-faulty client 
completes eventually, irrespective of the behavior of other clients.  
Most replication-based atomic memory algorithms guarantee liveness, as long as certain number of servers remain 
non-faulty; however, similar fault-tolerance levels can be achieved erasure-code based systems only in 
carefully designed algorithms. Moreover, there are some lower bound results on the  storage-cost for erasure-coded atomic memory algorithms in the presence of faulty clients~\cite{cadambe2016information, SCCK15}. As a result, to  circumvent this restriction many authors assume some restricting assumptions.  CASGC, ORCAS-A and ORCAS-B~\cite{DGL08} assume a bound on the number of concurrent operations. In the same vein, our protocol assume a known bound on the number of concurrent writes with a read to achieve liveness. 
